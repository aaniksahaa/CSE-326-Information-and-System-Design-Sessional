

\documentclass{article}
\usepackage[a4paper, margin=2cm]{geometry}
\usepackage{hyperref}
\usepackage{longtable}
\usepackage{graphicx}
\usepackage{booktabs}
\usepackage[label=corner]{karnaugh-map}
\usepackage{tabularray}
\usepackage{multicol}
\usepackage{multirow}
\usepackage{array}
\usepackage{hyperref}
\hypersetup{
    colorlinks=true,
    linkcolor=blue,
    }
\usepackage{float}
\usepackage{caption}
\usepackage{subcaption}
\usepackage[siunitx, RPvoltages]{circuitikz}
\usetikzlibrary{calc}
\usepackage{tikz, pgfplots}
\usetikzlibrary{positioning}
\usepackage{enumitem}
\usepackage{biblatex}
\usepackage{rotating}
\usepackage{ragged2e}
\usepackage[outputdir=../../]{minted}


\title{CSE 306 \\
Computer Architecture Sessional \\

\vspace{5mm}

\begin{figure}[h]
    \centering
    \includegraphics[width=0.3\textwidth]{images/buet.png}
    \label{fig:enter-label}
\end{figure}

\vspace{5mm}
Assignment-3\\
4-bit MIPS Implementation \\
\vspace{10mm}
Section - A1 \\
Group - 01 \\
\vspace{15mm}
\RaggedRight
Group Members: \\
\normalsize	{
\centering
\begin{enumerate}[]
    \item 2005001 - Anik Saha
    \item 2005012 - Abrar Jahin Sarker
    \item 2005013 - Al Muhit Muhtadi
    \item 2005017 - Abdullah Muhammed Amimul Ehsan
    \item 2005023 - Jaber Ahmed Deeder
\end{enumerate}
}
}
\author{}
\date{}
    
    \begin{document}

\maketitle

\newpage

\section{\textbf{Workspace Management}}
\subsection{\textbf{Use Case: Creation of New Workspace}}
\textbf{Scenario:} User creates a new workspace from their account.\\
\textbf{Actors:} User\\
\textbf{Steps:}
\begin{enumerate}
\item User navigates to 'Settings & Members' in the sidebar.
\item User selects 'Create or Join Workspace'.
\item User enters the workspace name and email domain, if applicable.
\item User clicks 'Create' to complete the process.
\end{enumerate}
\textbf{Test Cases}

            \begin{longtable}{|p{0.3\textwidth}|p{0.6\textwidth}|}
            \hline
            \textbf{Name} & \textbf{Description} \\
            \hline
            Successful Workspace Creation & Verify that the user can successfully create a new workspace with a unique name. \\
\hline
Duplicate Workspace Name & Verify that creating a workspace with an existing name does not matter \\
\hline
Exceeding Workspace Limit & Verify that the user cannot create more workspaces than their plan allows. \\
\hline
\end{longtable}\subsection{\textbf{Use Case: Switching Workspaces}}
\textbf{Scenario:} User switches between multiple workspaces they are a part of.\\
\textbf{Actors:} User\\
\textbf{Steps:}
\begin{enumerate}
\item User clicks on their profile picture in the top-left corner.
\item User selects the workspace they want to switch to from the dropdown menu.
\item User is redirected to the selected workspace.
\end{enumerate}
\textbf{Test Cases}

            \begin{longtable}{|p{0.3\textwidth}|p{0.6\textwidth}|}
            \hline
            \textbf{Name} & \textbf{Description} \\
            \hline
            Successful Workspace Switch & Verify that the user can successfully switch to another workspace. \\
\hline
Switch to Unavailable Workspace & Verify that the user cannot switch to a workspace they have left or been removed from. \\
\hline
Switch to Same Workspace & Verify that selecting the current workspace does not reload or cause any issues. \\
\hline
\end{longtable}\subsection{\textbf{Use Case: Joining an Existing Workspace}}
\textbf{Scenario:} User joins an existing workspace by invitation or allowed domain.\\
\textbf{Actors:} User\\
\textbf{Steps:}
\begin{enumerate}
\item User clicks on the invitation link or logs in with an email matching the allowed domain.
\item User accepts the invitation to join the workspace.
\end{enumerate}
\textbf{Test Cases}

            \begin{longtable}{|p{0.3\textwidth}|p{0.6\textwidth}|}
            \hline
            \textbf{Name} & \textbf{Description} \\
            \hline
            Successful Join via Invitation & Verify that the user can join the workspace successfully using an invitation link. \\
\hline
Join with Invalid Invitation & Verify that the user cannot join the workspace using an expired or invalid invitation link. \\
\hline
Join via Allowed Domain & Verify that the user can join the workspace using an email with an allowed domain. \\
\hline
Join via Restricted Domain & Verify that the user cannot join the workspace using an email with a restricted domain. \\
\hline
\end{longtable}\subsection{\textbf{Use Case: Update Workspace Name}}
\textbf{Scenario:} User updates the name of current workspace\\
\textbf{Actors:} User\\
\textbf{Steps:}
\begin{enumerate}
\item User enters workspace name
\end{enumerate}
\textbf{Test Cases}

            \begin{longtable}{|p{0.3\textwidth}|p{0.6\textwidth}|}
            \hline
            \textbf{Name} & \textbf{Description} \\
            \hline
            Valid Workspace Name Selection & Verify that the user can successfully update the workspace name with a valid input. \\
\hline
Empty Workspace Name & Verify that the user cannot update the workspace name with an empty input. \\
\hline
\end{longtable}\subsection{\textbf{Use Case: Upload Workspace Icon}}
\textbf{Scenario:} User updates the icon of current workspace\\
\textbf{Actors:} User\\
\textbf{Steps:}
\begin{enumerate}
\item User selects workspace icon
\end{enumerate}
\textbf{Test Cases}

            \begin{longtable}{|p{0.3\textwidth}|p{0.6\textwidth}|}
            \hline
            \textbf{Name} & \textbf{Description} \\
            \hline
            Valid Workspace Icon Selection & Verify that the user can successfully update the workspace icon with a valid file. \\
\hline
Empty Workspace Icon & Verify that the user cannot update the workspace icon without selecting a file. \\
\hline
Invalid Workspace Icon File Type & Verify that the user cannot update the workspace icon with an invalid file type. \\
\hline
Corrupted Workspace Icon & Verify that the user cannot update the workspace icon with a valid file type but corrupted data. \\
\hline
Large Workspace Icon & Verify that the user cannot update the workspace icon with a file that exceeds the maximum file size. \\
\hline
\end{longtable}\subsection{\textbf{Use Case: Delete Workspace}}
\textbf{Scenario:} User deletes the current workspace\\
\textbf{Actors:} User\\
\textbf{Steps:}
\begin{enumerate}
\item User types 'Workspace Name' confirms deletion
\end{enumerate}
\textbf{Test Cases}

            \begin{longtable}{|p{0.3\textwidth}|p{0.6\textwidth}|}
            \hline
            \textbf{Name} & \textbf{Description} \\
            \hline
            Correct Workspace Name & Verify that the user can successfully delete the workspace with the correct workspace name. \\
\hline
Incorrect Workspace Name & Verify that the user cannot delete the workspace with an incorrect workspace name. \\
\hline
\end{longtable}\subsection{\textbf{Use Case: Leaving a Workspace}}
\textbf{Scenario:} User leaves a workspace they are a member of.\\
\textbf{Actors:} User\\
\textbf{Steps:}
\begin{enumerate}
\item User navigates to 'Settings & Members' in the sidebar.
\item User clicks on the 'Leave Workspace' button.
\item User confirms the action in the dialog box.
\end{enumerate}
\textbf{Test Cases}

            \begin{longtable}{|p{0.3\textwidth}|p{0.6\textwidth}|}
            \hline
            \textbf{Name} & \textbf{Description} \\
            \hline
            Successful Workspace Leave & Verify that the user can successfully leave a workspace they are a member of. \\
\hline
Leave Non-Member Workspace & Verify that the user cannot leave a workspace they are not a member of. \\
\hline
Cancel Leave Workspace & Verify that the user can cancel the leave action in the confirmation dialog. \\
\hline
\end{longtable}\subsection{\textbf{Use Case: Adding Another Account}}
\textbf{Scenario:} User adds another email account to their Notion account.\\
\textbf{Actors:} User\\
\textbf{Steps:}
\begin{enumerate}
\item User navigates to 'Settings & Members' in the sidebar.
\item User clicks on 'My Account' and selects 'Add Email'.
\item User enters the new email address and verifies it through the email link sent.
\item User confirms the new email as added to their account.
\end{enumerate}
\textbf{Test Cases}

            \begin{longtable}{|p{0.3\textwidth}|p{0.6\textwidth}|}
            \hline
            \textbf{Name} & \textbf{Description} \\
            \hline
            Successful Account Addition & Verify that the user can successfully add another email account to their Notion account. \\
\hline
Invalid Email Format & Verify that the system prevents adding an email with an invalid format. \\
\hline
Email Already Linked & Verify that the system prevents adding an email that is already linked to the account. \\
\hline
Verify Email for Account Addition & Verify that the user must verify the email through the link sent to successfully add the account. \\
\hline
\end{longtable}\subsection{\textbf{Use Case: Import Workspace}}
\textbf{Scenario:} User imports entire workspace data from a file or cloud storage\\
\textbf{Actors:} User\\
\textbf{Steps:}
\begin{enumerate}
\item User uploads the file to be imported
\item Alternatively, user selects a cloud storage
\end{enumerate}
\textbf{Test Cases}

            \begin{longtable}{|p{0.3\textwidth}|p{0.6\textwidth}|}
            \hline
            \textbf{Name} & \textbf{Description} \\
            \hline
            Valid Workspace File & Verify that the user can successfully import the workspace with a valid file. \\
\hline
Invalid Workspace File & Verify that the user cannot import the workspace with an invalid file. \\
\hline
Valid Workspace Cloud Storage & Verify that the user can successfully import the workspace from a valid cloud storage. \\
\hline
Invalid Workspace Cloud Storage & Verify that the user cannot import the workspace from an invalid cloud storage. \\
\hline
\end{longtable}\subsection{\textbf{Use Case: Add Member to Workspace}}
\textbf{Scenario:} User adds a member to the current workspace\\
\textbf{Actors:} User\\
\textbf{Steps:}
\begin{enumerate}
\item User searches the email of the member to be added
\item User selects the person to be added or alternatively sends an invitation email
\end{enumerate}
\textbf{Test Cases}

            \begin{longtable}{|p{0.3\textwidth}|p{0.6\textwidth}|}
            \hline
            \textbf{Name} & \textbf{Description} \\
            \hline
            Invalid Email & Verify that the user cannot send an invitation to a member with an invalid email. \\
\hline
Person Already in Workspace & Verify that the user cannot add a person to the workspace who is already a member. \\
\hline
Person Found on Search & Verify that the user can successfully add a person to the workspace. \\
\hline
Person not Found on Search & Verify that the user can successfully send an invitation to a person to join the workspace. \\
\hline
Add Valid Member with Plus Subscription & Verify that the user can successfully add a member with the 'Member' role when the workspace has a Plus subscription. \\
\hline
Add Member Without Plus Subscription & Verify that when the user does not have a Plus subscription, all new members are assigned the 'Owner' role. \\
\hline
Add Member with Invalid Email & Verify that the system handles invalid email addresses during member addition. \\
\hline
Assign Workspace Owner Role & Verify that the user can assign the 'Workspace Owner' role to a new member regardless of subscription. \\
\hline
Invite Using Restricted Domain & Verify that the system prevents adding members with email addresses from restricted domains. \\
\hline
\end{longtable}\subsection{\textbf{Use Case: Managing Allowed Email Domains}}
\textbf{Scenario:} User sets or updates allowed email domains for the workspace.\\
\textbf{Actors:} User\\
\textbf{Steps:}
\begin{enumerate}
\item User navigates to 'Settings' under the workspace menu.
\item User scrolls to 'Allowed email domains'.
\item User enters the domain name to be allowed (e.g., 'company.com','ugrad.cse.buet.ac.bd').
\item User clicks 'Update' to save changes.
\end{enumerate}
\textbf{Test Cases}

            \begin{longtable}{|p{0.3\textwidth}|p{0.6\textwidth}|}
            \hline
            \textbf{Name} & \textbf{Description} \\
            \hline
            Adding Valid Allowed Domain & Verify that the user can add a valid email domain to the allowed list. \\
\hline
Adding Invalid Allowed Domain & Verify that the system handles invalid domain inputs correctly. \\
\hline
Adding Public Domain & Verify that the system handles public domain inputs correctly. \\
\hline
Adding Educational Domain & Verify that the system handles public domain inputs correctly. \\
\hline
Removal Allowed Domain & Verify that the user can remove a previously added domain from the allowed list. \\
\hline
\end{longtable}\subsection{\textbf{Use Case: Export Workspace Content}}
\textbf{Scenario:} User exports all workspace content for backup, sharing, or legal purposes.\\
\textbf{Actors:} User\\
\textbf{Steps:}
\begin{enumerate}
\item User navigates to 'Settings' under the workspace menu.
\item User clicks on 'Export all workspace content'.
\item User selects export options such as format (Markdown, HTML, CSV, PDF) and content inclusion.
\item User clicks 'Export' to download the content.
\end{enumerate}
\textbf{Test Cases}

            \begin{longtable}{|p{0.3\textwidth}|p{0.6\textwidth}|}
            \hline
            \textbf{Name} & \textbf{Description} \\
            \hline
            Valid Export as Markdown & CSV & Verify that the user can successfully export all workspace content in Markdown & CSV format. \\
\hline
Valid Export as HTML & Verify that the user can successfully export all workspace content in HTML format. \\
\hline
Valid Export as PDF with Enterprise Plan & Verify that the user can export workspace content as PDF when subscribed to an Enterprise plan. \\
\hline
PDF Export without Enterprise Plan & Verify that PDF export is not available without an Enterprise plan. \\
\hline
Export with Insufficient Permissions & Verify that members without export permissions cannot export workspace content. \\
\hline
Exclude Private Pages from Export & Verify that private pages of other users are not included in the export. \\
\hline
Incorrect Export Format & Verify that the user cannot export the workspace with incorrect workspace export format. \\
\hline
\end{longtable}\section{\textbf{Content Management}}
\subsection{\textbf{Use Case: Rich Text Formatting}}
\textbf{Scenario:} Users format their text with rich options adding links.\\
\textbf{Actors:} User\\
\textbf{Steps:}
\begin{enumerate}
\item User selects the text block.
\item User applies the formatting options
\item User checks the formatted result.
\end{enumerate}
\textbf{Test Cases}

            \begin{longtable}{|p{0.3\textwidth}|p{0.6\textwidth}|}
            \hline
            \textbf{Name} & \textbf{Description} \\
            \hline
            Valid Text Hyperlinking & Verify that a user can add a hyperlink to selected text. \\
\hline
Valid Markdown Syntax & Verify that correct markdown content is rendered successfully. \\
\hline
Invalid Markdown Syntax & Verify that incorrect markdown results in failure to apply formatting. \\
\hline
Invalid Text Hyperlinking & Verify that an invalid link is not added to a content text. \\
\hline
Valid font change & Verify that a user can select a valid font of selected text. \\
\hline
Valid Text Coloring & Verify that a user can apply a color to the text. \\
\hline
\end{longtable}\subsection{\textbf{Use Case: Organize Content into Columns}}
\textbf{Scenario:} A user arranges content side-by-side by creating multiple columns on a page.\\
\textbf{Actors:} User\\
\textbf{Steps:}
\begin{enumerate}
\item User drags and drops blocks into columns.
\item Content is organized into multiple columns.
\end{enumerate}
\textbf{Test Cases}

            \begin{longtable}{|p{0.3\textwidth}|p{0.6\textwidth}|}
            \hline
            \textbf{Name} & \textbf{Description} \\
            \hline
            Create Two Columns & Verify that a user can create two columns by dragging and dropping blocks. \\
\hline
Remove Columns & Verify that a user can remove a column by dragging the content back. \\
\hline
\end{longtable}\subsection{\textbf{Use Case: Add Headings for Structure}}
\textbf{Scenario:} The user adds headings to structure the content on a Notion page.\\
\textbf{Actors:} User\\
\textbf{Steps:}
\begin{enumerate}
\item User adds H1, H2, and H3 headings for different sections.
\item Content is structured with headings.
\end{enumerate}
\textbf{Test Cases}

            \begin{longtable}{|p{0.3\textwidth}|p{0.6\textwidth}|}
            \hline
            \textbf{Name} & \textbf{Description} \\
            \hline
            Valid Heading Creation & Verify that a user can add an heading to the page. \\
\hline
Header Creation in Incorrect Block & Verify that trying to add a header inside unsupported blocks (e.g., code blocks) fails. \\
\hline
\end{longtable}\subsection{\textbf{Use Case: Add Icons and Cover Art}}
\textbf{Scenario:} A user personalizes a page by adding icons and cover images.\\
\textbf{Actors:} User\\
\textbf{Steps:}
\begin{enumerate}
\item User adds or changes the page icon.
\item User adds or changes the cover image.
\end{enumerate}
\textbf{Test Cases}

            \begin{longtable}{|p{0.3\textwidth}|p{0.6\textwidth}|}
            \hline
            \textbf{Name} & \textbf{Description} \\
            \hline
            Valid Icon Selection & Verify that a user can successfully add an icon to a page. \\
\hline
Valid Cover Image & Verify that a user can successfully change the cover image on a page. \\
\hline
Invalid Cover Image type & Verify that an image change is rejected if the image type is wrong. \\
\hline
\end{longtable}\subsection{\textbf{Use Case: Add and Manage Images in Content}}
\textbf{Scenario:} A user uploads images, arranges them on a page, and resizes them as needed.\\
\textbf{Actors:} User\\
\textbf{Steps:}
\begin{enumerate}
\item User uploads an image to the page.
\item User arranges and resizes the image.
\item User adds captions and alt text for the image.
\end{enumerate}
\textbf{Test Cases}

            \begin{longtable}{|p{0.3\textwidth}|p{0.6\textwidth}|}
            \hline
            \textbf{Name} & \textbf{Description} \\
            \hline
            Valid Image Upload & Verify that a user can successfully upload an image to a page. \\
\hline
Valid Image Resize & Verify that a user can resize an uploaded image to a reasonable percentage. \\
\hline
Too Large Resize Percentage & Verify that an image resize is rejected when the entered percentage amount is too large. \\
\hline
\end{longtable}\subsection{\textbf{Use Case: Add and Embed Files}}
\textbf{Scenario:} A user uploads or embeds files like PDFs or documents into a Notion page.\\
\textbf{Actors:} User\\
\textbf{Steps:}
\begin{enumerate}
\item User uploads a PDF file to the page.
\item User embeds a file link from an external source.
\item User arranges the file block on the page.
\end{enumerate}
\textbf{Test Cases}

            \begin{longtable}{|p{0.3\textwidth}|p{0.6\textwidth}|}
            \hline
            \textbf{Name} & \textbf{Description} \\
            \hline
            Valid PDF Upload & Verify that a user can upload a PDF file to the page. \\
\hline
Too Large Filesize & Verify that a file that is too large is discarded. \\
\hline
Valid External File Embed & Verify that a user can embed a file link from an external URL. \\
\hline
Invalid External File Embed Link & Verify that a file embed is rejected when the link is invalid. \\
\hline
\end{longtable}\subsection{\textbf{Use Case: Embed and Manage Media}}
\textbf{Scenario:} A user embeds videos and audio files into a Notion page.\\
\textbf{Actors:} User\\
\textbf{Steps:}
\begin{enumerate}
\item User embeds a video from a streaming service.
\item User uploads an audio file.
\item User resizes and aligns the media blocks.
\end{enumerate}
\textbf{Test Cases}

            \begin{longtable}{|p{0.3\textwidth}|p{0.6\textwidth}|}
            \hline
            \textbf{Name} & \textbf{Description} \\
            \hline
            Embed Video & Verify that a user can embed a video from a streaming platform. \\
\hline
Upload Audio File & Verify that a user can upload an audio file and play it in Notion. \\
\hline
Unsupported Media Format & Verify that unsupported audio or video formats fail to play. \\
\hline
Exceeds File Size Limit & Verify that uploading a file that exceeds the size limit fails. \\
\hline
\end{longtable}\subsection{\textbf{Use Case: Link to Notion Page}}
\textbf{Scenario:} A user links one Notion page to another within a paragraph or as a block.\\
\textbf{Actors:} User\\
\textbf{Steps:}
\begin{enumerate}
\item User types '@', '[[' or '+' followed by the page name.
\item User selects the page from the dropdown.
\item Link is added in the desired format.
\end{enumerate}
\textbf{Test Cases}

            \begin{longtable}{|p{0.3\textwidth}|p{0.6\textwidth}|}
            \hline
            \textbf{Name} & \textbf{Description} \\
            \hline
            Valid Page Linking in Paragraph & Verify that a user can link another Notion page inline within a paragraph. \\
\hline
Valid Page linking as Block & Verify that a user can add a Notion page link as a block. \\
\hline
Broken Link Creation & Verify that trying to create a link to a deleted or non-existent page results in an error. \\
\hline
\end{longtable}\subsection{\textbf{Use Case: Link to a Web Page}}
\textbf{Scenario:} A user pastes a URL into Notion and formats it as a mention for easier readability.\\
\textbf{Actors:} User\\
\textbf{Steps:}
\begin{enumerate}
\item User pastes the URL into a Notion page.
\item User selects 'Paste as mention' to format the link.
\end{enumerate}
\textbf{Test Cases}

            \begin{longtable}{|p{0.3\textwidth}|p{0.6\textwidth}|}
            \hline
            \textbf{Name} & \textbf{Description} \\
            \hline
            Valid Web Link Embedding & Verify that a user can embed a web link and format it as a mention. \\
\hline
Broken Web Link & Verify that a broken link is detected by the system and rejected. \\
\hline
\end{longtable}\subsection{\textbf{Use Case: Transfer Content to Another Account}}
\textbf{Scenario:} A user transfers ownership of content (pages, databases) from one Notion account to another.\\
\textbf{Actors:} User, New Account Owner\\
\textbf{Steps:}
\begin{enumerate}
\item User shares the page with the new account.
\item User grants ownership permissions to the new account.
\end{enumerate}
\textbf{Test Cases}

            \begin{longtable}{|p{0.3\textwidth}|p{0.6\textwidth}|}
            \hline
            \textbf{Name} & \textbf{Description} \\
            \hline
            Valid Ownership Transfer & Verify that a user can transfer ownership of a page to another account. \\
\hline
Transfer Between Incompatible Accounts & Verify that transferring content between a free plan and a paid account fails if file size limits are exceeded. \\
\hline
\end{longtable}\subsection{\textbf{Use Case: Duplicate and Delete Content}}
\textbf{Scenario:} A user duplicates or deletes content on a Notion page.\\
\textbf{Actors:} User\\
\textbf{Steps:}
\begin{enumerate}
\item User selects a block of content.
\item User chooses either 'Duplicate' or 'Delete' from the menu.
\end{enumerate}
\textbf{Test Cases}

            \begin{longtable}{|p{0.3\textwidth}|p{0.6\textwidth}|}
            \hline
            \textbf{Name} & \textbf{Description} \\
            \hline
            Duplicate Block & Verify that a user can duplicate a block of content. \\
\hline
Delete Page & Verify that a user can delete an entire page. \\
\hline
\end{longtable}\subsection{\textbf{Use Case: Restore Deleted Content}}
\textbf{Scenario:} A user restores deleted content from the trash within 30 days.\\
\textbf{Actors:} User\\
\textbf{Steps:}
\begin{enumerate}
\item User navigates to the trash.
\item User selects the deleted page to restore.
\end{enumerate}
\textbf{Test Cases}

            \begin{longtable}{|p{0.3\textwidth}|p{0.6\textwidth}|}
            \hline
            \textbf{Name} & \textbf{Description} \\
            \hline
            Restore Deleted Page & Verify that a user can restore a deleted page from the trash. \\
\hline
Restore Deleted Page After Timeout & Verify that restoring a deleted page fails after the allowed restoration time has passed. \\
\hline
\end{longtable}\subsection{\textbf{Use Case: Sync Content Across Pages}}
\textbf{Scenario:} A user syncs content across multiple Notion pages.\\
\textbf{Actors:} User\\
\textbf{Steps:}
\begin{enumerate}
\item User copies the content from a page.
\item User pastes it on a different page with the 'Paste and sync' option.
\item Content is now synced across pages.
\end{enumerate}
\textbf{Test Cases}

            \begin{longtable}{|p{0.3\textwidth}|p{0.6\textwidth}|}
            \hline
            \textbf{Name} & \textbf{Description} \\
            \hline
            Sync Content Successfully & Verify that the user can sync content across two pages. \\
\hline
Sync Content Failure - No Permissions & Verify that a user cannot sync content to a page where they don't have permission. \\
\hline
\end{longtable}\subsection{\textbf{Use Case: Unsync Content}}
\textbf{Scenario:} A user unsyncs specific content across pages.\\
\textbf{Actors:} User\\
\textbf{Steps:}
\begin{enumerate}
\item User selects the synced block.
\item User chooses the 'Unsync' option to remove the sync.
\end{enumerate}
\textbf{Test Cases}

            \begin{longtable}{|p{0.3\textwidth}|p{0.6\textwidth}|}
            \hline
            \textbf{Name} & \textbf{Description} \\
            \hline
            Unsync Specific Block & Verify that a user can unsync a specific copy of the block. \\
\hline
Unsync Failure - Not Original Block & Verify that unsyncing fails when attempting to unsync a block that is not the original. \\
\hline
\end{longtable}\subsection{\textbf{Use Case: Create and Edit Code Blocks}}
\textbf{Scenario:} User creates and edits code blocks within Notion for documentation or sharing code snippets.\\
\textbf{Actors:} User\\
\textbf{Steps:}
\begin{enumerate}
\item User navigates to a Notion page and selects the option to add a new block.
\item User selects the code block option from the available block types.
\item User types or pastes code into the code block.
\item User modifies the code block by selecting the language, formatting, or editing the code.
\end{enumerate}
\textbf{Test Cases}

            \begin{longtable}{|p{0.3\textwidth}|p{0.6\textwidth}|}
            \hline
            \textbf{Name} & \textbf{Description} \\
            \hline
            Valid Code Block Creation & Verify that a user can successfully create a code block and input code. \\
\hline
Valid Language Selection & Verify that a user can select a programming language for a code block. \\
\hline
Invalid Language Selection & Verify that a user can select a programming language for a code block. \\
\hline
Empty Code Block & Verify that the system handles an empty code block without any code input. \\
\hline
Valid Code Block Editing & Verify that a user can successfully edit an existing code block. \\
\hline
\end{longtable}\subsection{\textbf{Use Case: Create and Display Math Equations in Notion}}
\textbf{Scenario:} A user creates and formats math equations in Notion using LaTeX syntax.\\
\textbf{Actors:} User\\
\textbf{Steps:}
\begin{enumerate}
\item User types an equation using the inline math command `$$` or the block equation command `/math`.
\item The system processes the LaTeX syntax.
\item The system displays the equation in the correct format.
\end{enumerate}
\textbf{Test Cases}

            \begin{longtable}{|p{0.3\textwidth}|p{0.6\textwidth}|}
            \hline
            \textbf{Name} & \textbf{Description} \\
            \hline
            Valid Inline Equation & Verify that a valid inline math equation is rendered correctly. \\
\hline
Invalid Inline Equation & Verify that the system handles improper LaTeX syntax in inline equations. \\
\hline
Valid Block Equation & Verify that a block equation is correctly rendered in Notion. \\
\hline
Invalid Block Equation & Verify that the system handles incorrect LaTeX syntax in block equations. \\
\hline
\end{longtable}\section{\textbf{Database Creation and Management}}
\subsection{\textbf{Use Case: Creating a Database}}
\textbf{Scenario:} User creates a new database in Notion.\\
\textbf{Actors:} User\\
\textbf{Steps:}
\begin{enumerate}
\item User navigates to a Notion page and types '/' to access the command menu.
\item User selects a database type (e.g., table, board, list).
\item User enters the name of the new database.
\item User configures properties for the database.
\end{enumerate}
\textbf{Test Cases}

            \begin{longtable}{|p{0.3\textwidth}|p{0.6\textwidth}|}
            \hline
            \textbf{Name} & \textbf{Description} \\
            \hline
            Creating a Table Database & Verify that the user can create a new table database. \\
\hline
Creating a Database Without Name & Verify that the user cannot create a database without providing a name. \\
\hline
Creating a Database with Existing Name & Verify that the system handles duplicate database names within the same workspace. \\
\hline
\end{longtable}\subsection{\textbf{Use Case: Customizing Database Views}}
\textbf{Scenario:} User customizes the views of a database to display data in different formats.\\
\textbf{Actors:} User\\
\textbf{Steps:}
\begin{enumerate}
\item User navigates to the database and selects 'Add a view'.
\item User chooses a view type such as 'Table', 'Board', 'Calendar', etc.
\item User customizes the view by adding filters, sorts, and grouping.
\item User saves the view for future use.
\end{enumerate}
\textbf{Test Cases}

            \begin{longtable}{|p{0.3\textwidth}|p{0.6\textwidth}|}
            \hline
            \textbf{Name} & \textbf{Description} \\
            \hline
            Creating a Kanban Board View & Verify that the user can create a board view to display tasks by status. \\
\hline
Applying Filters to Calendar View & Verify that the user can apply filters to show only events for a specific team in the calendar view. \\
\hline
Sorting Table View by Priority & Verify that the user can sort tasks by priority in the table view. \\
\hline
\end{longtable}\subsection{\textbf{Use Case: Managing Database Pages}}
\textbf{Scenario:} User manages individual pages within a database, adding, opening, and editing them.\\
\textbf{Actors:} User\\
\textbf{Steps:}
\begin{enumerate}
\item User navigates to the database and clicks on 'New' to add a new page.
\item User enters the details for the new page and saves it.
\item User opens an existing page to edit its content.
\item User modifies properties or adds new content to the page.
\end{enumerate}
\textbf{Test Cases}

            \begin{longtable}{|p{0.3\textwidth}|p{0.6\textwidth}|}
            \hline
            \textbf{Name} & \textbf{Description} \\
            \hline
            Creating a New Database Page & Verify that the user can create a new page within the database. \\
\hline
Editing Existing Page Content & Verify that the user can edit an existing page's content in the database. \\
\hline
Deleting a Database Page & Verify that the user can delete a page from the database. \\
\hline
\end{longtable}\subsection{\textbf{Use Case: Collaborating in a Database}}
\textbf{Scenario:} User collaborates with team members in a shared database with specific permissions.\\
\textbf{Actors:} User\\
\textbf{Steps:}
\begin{enumerate}
\item User shares the database with team members with 'Can edit content' permissions.
\item Team members add and edit pages within the database.
\item User reviews changes made by team members and provides feedback in comments.
\end{enumerate}
\textbf{Test Cases}

            \begin{longtable}{|p{0.3\textwidth}|p{0.6\textwidth}|}
            \hline
            \textbf{Name} & \textbf{Description} \\
            \hline
            Granting Edit Permissions & Verify that the user can grant 'Can edit content' permissions to team members. \\
\hline
Editing Content with Edit Permissions & Verify that team members with 'Can edit content' permissions can add and edit pages in the database. \\
\hline
Restricting Property Editing & Verify that team members cannot edit properties or views with 'Can edit content' permissions. \\
\hline
\end{longtable}\subsection{\textbf{Use Case: Managing Database Lock}}
\textbf{Scenario:} User locks and unlocks a database to control structure changes.\\
\textbf{Actors:} User\\
\textbf{Steps:}
\begin{enumerate}
\item User navigates to the database settings.
\item User selects 'Lock/Unlock database' and confirms the action.
\end{enumerate}
\textbf{Test Cases}

            \begin{longtable}{|p{0.3\textwidth}|p{0.6\textwidth}|}
            \hline
            \textbf{Name} & \textbf{Description} \\
            \hline
            Lock and Unlock Database & Verify that the user can lock and unlock a database, preventing or allowing structural changes. \\
\hline
Lock Database Without Permission & Verify that users without permissions cannot lock/unlock the database. \\
\hline
\end{longtable}\subsection{\textbf{Use Case: Handling Changes in a Locked Database}}
\textbf{Scenario:} User attempts to change structure of a locked database.\\
\textbf{Actors:} User\\
\textbf{Steps:}
\begin{enumerate}
\item User tries to modify properties or views of the locked database.
\end{enumerate}
\textbf{Test Cases}

            \begin{longtable}{|p{0.3\textwidth}|p{0.6\textwidth}|}
            \hline
            \textbf{Name} & \textbf{Description} \\
            \hline
            Edit Locked Database Properties or Views & Verify that users cannot modify properties, add views, or delete properties in a locked database. \\
\hline
\end{longtable}\subsection{\textbf{Use Case: Managing Database Properties}}
\textbf{Scenario:} User adds, edits, and deletes properties in a database.\\
\textbf{Actors:} User\\
\textbf{Steps:}
\begin{enumerate}
\item User navigates to database settings.
\item User selects 'New property' to add a property.
\item User edits or deletes properties as needed.
\end{enumerate}
\textbf{Test Cases}

            \begin{longtable}{|p{0.3\textwidth}|p{0.6\textwidth}|}
            \hline
            \textbf{Name} & \textbf{Description} \\
            \hline
            Adding a New Property & Verify that the user can add a new property with a valid name and type. \\
\hline
Adding a Property with Empty Name & Verify that the user cannot add a property with an empty name. \\
\hline
Deleting a Property & Verify that the user can delete a property. \\
\hline
\end{longtable}\subsection{\textbf{Use Case: Editing Property Values}}
\textbf{Scenario:} User edits values for different property types in the database.\\
\textbf{Actors:} User\\
\textbf{Steps:}
\begin{enumerate}
\item User selects a property cell in the database.
\item User enters or updates the value for the property.
\item User saves the changes.
\end{enumerate}
\textbf{Test Cases}

            \begin{longtable}{|p{0.3\textwidth}|p{0.6\textwidth}|}
            \hline
            \textbf{Name} & \textbf{Description} \\
            \hline
            Editing Text Property & Verify that the user can enter text into a text property. \\
\hline
Entering Invalid Number in Number Property & Verify that the user cannot enter non-numeric values in a number property. \\
\hline
Clearing a Date Property & Verify that the user can clear the value in a date property. \\
\hline
\end{longtable}\subsection{\textbf{Use Case: Handling Invalid and Empty Values for Properties}}
\textbf{Scenario:} User enters invalid or empty values for different property types in a database.\\
\textbf{Actors:} User\\
\textbf{Steps:}
\begin{enumerate}
\item User selects a property cell in the database.
\item User attempts to enter invalid or empty values for each property type.
\item System validates and either accepts or rejects the input.
\end{enumerate}
\textbf{Test Cases}

            \begin{longtable}{|p{0.3\textwidth}|p{0.6\textwidth}|}
            \hline
            \textbf{Name} & \textbf{Description} \\
            \hline
            Entering Invalid Number & Verify that the system rejects non-numeric values for a number property. \\
\hline
Empty Number Property & Verify that the system accepts an empty value for a number property. \\
\hline
Invalid Email Format & Verify that the system rejects an invalid email format. \\
\hline
Empty Email Property & Verify that the system accepts an empty value for an email property. \\
\hline
Invalid URL Format & Verify that the system rejects an invalid URL format. \\
\hline
Empty URL Property & Verify that the system accepts an empty value for a URL property. \\
\hline
Invalid Date Format & Verify that the system rejects an invalid date format. \\
\hline
Empty Date Property & Verify that the system accepts an empty value for a date property. \\
\hline
Invalid Phone Format & Verify that the system rejects an invalid phone number format. \\
\hline
Empty Phone Property & Verify that the system accepts an empty value for a phone property. \\
\hline
\end{longtable}\subsection{\textbf{Use Case: Creating and Managing Relations}}
\textbf{Scenario:} User creates and manages relations between databases to link related data.\\
\textbf{Actors:} User\\
\textbf{Steps:}
\begin{enumerate}
\item User navigates to the database settings.
\item User adds a new 'Relation' property.
\item User selects the target database to link.
\item User configures relation settings and saves.
\end{enumerate}
\textbf{Test Cases}

            \begin{longtable}{|p{0.3\textwidth}|p{0.6\textwidth}|}
            \hline
            \textbf{Name} & \textbf{Description} \\
            \hline
            Creating a One-Way Relation & Verify that the user can create a one-way relation between two databases. \\
\hline
Creating a Two-Way Relation & Verify that the user can create a two-way relation between two databases. \\
\hline
Relating a Database to Itself & Verify that the user can create a relation within the same database to link related items. \\
\hline
\end{longtable}\subsection{\textbf{Use Case: Creating and Managing Rollups}}
\textbf{Scenario:} User creates rollup properties to aggregate data from related databases.\\
\textbf{Actors:} User\\
\textbf{Steps:}
\begin{enumerate}
\item User navigates to the database settings.
\item User adds a new 'Rollup' property.
\item User selects the related property to roll up and chooses a calculation method.
\item User saves the rollup property settings.
\end{enumerate}
\textbf{Test Cases}

            \begin{longtable}{|p{0.3\textwidth}|p{0.6\textwidth}|}
            \hline
            \textbf{Name} & \textbf{Description} \\
            \hline
            Creating a Sum Rollup & Verify that the user can create a rollup property to sum numeric values from related items. \\
\hline
Creating a Count Rollup & Verify that the user can create a rollup property to count non-empty values from related items. \\
\hline
Creating a Date Rollup & Verify that the user can create a rollup property to find the latest date from related items. \\
\hline
\end{longtable}\subsection{\textbf{Use Case: Handling Invalid or Empty Rollup Values}}
\textbf{Scenario:} User tries to create rollup properties with invalid or empty values.\\
\textbf{Actors:} User\\
\textbf{Steps:}
\begin{enumerate}
\item User attempts to create a rollup with invalid data types or empty values.
\end{enumerate}
\textbf{Test Cases}

            \begin{longtable}{|p{0.3\textwidth}|p{0.6\textwidth}|}
            \hline
            \textbf{Name} & \textbf{Description} \\
            \hline
            Invalid Data Type for Rollup & Verify that the system rejects rollup creation if the selected property is not compatible with the calculation. \\
\hline
Empty Values in Rollup Property & Verify that the system can handle rollup calculations with empty values correctly. \\
\hline
\end{longtable}\subsection{\textbf{Use Case: Managing Database Views}}
\textbf{Scenario:} User creates, customizes, and switches between different views in a database.\\
\textbf{Actors:} User\\
\textbf{Steps:}
\begin{enumerate}
\item User navigates to the database and selects 'Add a view'.
\item User chooses a view type such as 'Table', 'Board', 'Calendar', etc.
\item User customizes the view settings (properties, filters, sorts).
\item User switches between different views using the dropdown menu.
\end{enumerate}
\textbf{Test Cases}

            \begin{longtable}{|p{0.3\textwidth}|p{0.6\textwidth}|}
            \hline
            \textbf{Name} & \textbf{Description} \\
            \hline
            Creating a New View & Verify that the user can create a new view with customized settings. \\
\hline
Switching Between Views & Verify that the user can switch between different database views without errors. \\
\hline
Editing View Components & Verify that the user can edit existing view components such as layout and properties. \\
\hline
\end{longtable}\subsection{\textbf{Use Case: Applying Filters to a Database}}
\textbf{Scenario:} User applies filters to a database view to display specific data.\\
\textbf{Actors:} User\\
\textbf{Steps:}
\begin{enumerate}
\item User navigates to the database and clicks 'Filter'.
\item User selects a property and sets criteria for filtering.
\item User adds additional filters or groups using 'AND'/'OR' logic.
\item User saves the filter configuration.
\end{enumerate}
\textbf{Test Cases}

            \begin{longtable}{|p{0.3\textwidth}|p{0.6\textwidth}|}
            \hline
            \textbf{Name} & \textbf{Description} \\
            \hline
            Applying a Simple Filter & Verify that the user can apply a simple filter based on a single property. \\
\hline
Creating an Advanced Filter & Verify that the user can create an advanced filter using 'AND'/'OR' logic. \\
\hline
Removing a Filter & Verify that the user can remove an applied filter from the database view. \\
\hline
\end{longtable}\subsection{\textbf{Use Case: Sorting Items in a Database}}
\textbf{Scenario:} User sorts database items based on a specific property in ascending or descending order.\\
\textbf{Actors:} User\\
\textbf{Steps:}
\begin{enumerate}
\item User navigates to the database and clicks 'Sort'.
\item User selects a property and chooses ascending or descending order.
\item User adds multiple sorts if needed and adjusts their order.
\item User saves the sort configuration.
\end{enumerate}
\textbf{Test Cases}

            \begin{longtable}{|p{0.3\textwidth}|p{0.6\textwidth}|}
            \hline
            \textbf{Name} & \textbf{Description} \\
            \hline
            Sorting by Due Date & Verify that the user can sort items by 'Due Date' in ascending order. \\
\hline
Sorting by Priority and Status & Verify that the user can sort items by 'Priority' and 'Status' with different orders. \\
\hline
Removing a Sort & Verify that the user can remove an applied sort from the database view. \\
\hline
\end{longtable}\subsection{\textbf{Use Case: Grouping Items in a Database}}
\textbf{Scenario:} User groups database items by a specific property to organize data visually.\\
\textbf{Actors:} User\\
\textbf{Steps:}
\begin{enumerate}
\item User navigates to the database and clicks 'Group'.
\item User selects a property to group by (e.g., 'Status').
\item User arranges the groups in a desired order.
\item User saves the group configuration.
\end{enumerate}
\textbf{Test Cases}

            \begin{longtable}{|p{0.3\textwidth}|p{0.6\textwidth}|}
            \hline
            \textbf{Name} & \textbf{Description} \\
            \hline
            Grouping by Status & Verify that the user can group items by 'Status' in a board view. \\
\hline
Hiding Empty Groups & Verify that the user can hide groups without items in the database view. \\
\hline
Removing Grouping & Verify that the user can remove grouping from the database view. \\
\hline
\end{longtable}\subsection{\textbf{Use Case: Searching a Database}}
\textbf{Scenario:} User searches for specific items within a database using keywords.\\
\textbf{Actors:} User\\
\textbf{Steps:}
\begin{enumerate}
\item User navigates to the database and clicks the search icon.
\item User enters keywords related to page titles or properties.
\item User reviews the search results displayed in real-time.
\end{enumerate}
\textbf{Test Cases}

            \begin{longtable}{|p{0.3\textwidth}|p{0.6\textwidth}|}
            \hline
            \textbf{Name} & \textbf{Description} \\
            \hline
            Searching by Keyword & Verify that the user can search for items using a keyword. \\
\hline
No Matching Results & Verify that no items are displayed when there are no matching results. \\
\hline
Clearing Search & Verify that the user can clear the search input to return to the default view. \\
\hline
\end{longtable}\subsection{\textbf{Use Case: Managing Data in Table View}}
\textbf{Scenario:} User creates, modifies, and deletes rows and columns in a table view.\\
\textbf{Actors:} User\\
\textbf{Steps:}
\begin{enumerate}
\item User navigates to the database table view.
\item User adds, edits, or deletes rows and columns as needed.
\end{enumerate}
\textbf{Test Cases}

            \begin{longtable}{|p{0.3\textwidth}|p{0.6\textwidth}|}
            \hline
            \textbf{Name} & \textbf{Description} \\
            \hline
            Adding Rows & Verify that the user can add a new row to the table with all columns populated. \\
\hline
Deleting Columns & Verify that the user can delete a column and all associated data is removed. \\
\hline
Empty Values & Verify that leaving a cell empty in a required column triggers a validation message. \\
\hline
\end{longtable}\subsection{\textbf{Use Case: Organizing Simple Data in List View}}
\textbf{Scenario:} User organizes notes or articles using list view with minimal properties.\\
\textbf{Actors:} User\\
\textbf{Steps:}
\begin{enumerate}
\item User navigates to the database list view.
\item User adds, edits, or deletes list items as needed.
\end{enumerate}
\textbf{Test Cases}

            \begin{longtable}{|p{0.3\textwidth}|p{0.6\textwidth}|}
            \hline
            \textbf{Name} & \textbf{Description} \\
            \hline
            Adding Items & Verify that the user can add a new item with minimal properties like title and date. \\
\hline
Invalid Date & Verify that entering an invalid date format shows an error message. \\
\hline
Editing Item Title & Verify that the user can successfully edit the title of a list item. \\
\hline
\end{longtable}\subsection{\textbf{Use Case: Visualizing Workflows in Board View}}
\textbf{Scenario:} User creates boards to manage project stages or tasks.\\
\textbf{Actors:} User\\
\textbf{Steps:}
\begin{enumerate}
\item User navigates to the database board view.
\item User moves items between columns to represent changes in workflow stages.
\end{enumerate}
\textbf{Test Cases}

            \begin{longtable}{|p{0.3\textwidth}|p{0.6\textwidth}|}
            \hline
            \textbf{Name} & \textbf{Description} \\
            \hline
            Moving Items Across Stages & Verify that the user can drag and drop items between different stages (columns). \\
\hline
Invalid Stage Name & Verify that the system rejects a stage name with special characters. \\
\hline
Archiving Columns & Verify that the user can hide or archive columns with completed tasks. \\
\hline
\end{longtable}\subsection{\textbf{Use Case: Managing Events in Calendar View}}
\textbf{Scenario:} User schedules and views events on a calendar.\\
\textbf{Actors:} User\\
\textbf{Steps:}
\begin{enumerate}
\item User navigates to the database calendar view.
\item User adds, edits, or deletes events as needed.
\end{enumerate}
\textbf{Test Cases}

            \begin{longtable}{|p{0.3\textwidth}|p{0.6\textwidth}|}
            \hline
            \textbf{Name} & \textbf{Description} \\
            \hline
            Adding Multi-Day Events & Verify that the user can add events that span multiple days. \\
\hline
Invalid Date Range & Verify that entering a start date later than the end date triggers an error. \\
\hline
Changing Calendar Start Day & Verify that changing the week start day updates the view correctly. \\
\hline
\end{longtable}\subsection{\textbf{Use Case: Assigning Unique Identifiers to Database Items}}
\textbf{Scenario:} User assigns unique identifiers to each item for easy reference.\\
\textbf{Actors:} User\\
\textbf{Steps:}
\begin{enumerate}
\item User navigates to the database settings.
\item User configures or updates the unique ID format for each new item.
\end{enumerate}
\textbf{Test Cases}

            \begin{longtable}{|p{0.3\textwidth}|p{0.6\textwidth}|}
            \hline
            \textbf{Name} & \textbf{Description} \\
            \hline
            Generating Unique ID & Verify that each new item in the database is assigned a unique identifier automatically. \\
\hline
Custom ID Format & Verify that the user can set a custom format for unique IDs (e.g., prefix or suffix). \\
\hline
Duplicate ID Error & Verify that trying to manually set a duplicate ID triggers an error. \\
\hline
\end{longtable}\section{\textbf{Sharing and Collaboration}}
\subsection{\textbf{Use Case: Create New Teamspace}}
\textbf{Scenario:} A user creates a teamspace and invites other members by mentioning their emails. Those who accept the invitations become members of the teamspace.\\
\textbf{Actors:} User\\
\textbf{Steps:}
\begin{enumerate}
\item User provides name of the new Teamspace
\item User provides icon of the new Teamspace
\item User provides emails of other invited members
\end{enumerate}
\textbf{Test Cases}

            \begin{longtable}{|p{0.3\textwidth}|p{0.6\textwidth}|}
            \hline
            \textbf{Name} & \textbf{Description} \\
            \hline
            Valid Teamspace Creation & Verify that a user can successfully create a teamspace when he provides valid information. \\
\hline
Empty Teamspace Name & Verify that the system handles the case where a user provides empty teamspace name. \\
\hline
Invalid Icon Type & Verify that the system rejects unsupported file types for the teamspace icon. \\
\hline
Too Many Invited Emails & Verify that the system prevents a user under Free plan from inviting more than 10 guests in a single teamspace creation. \\
\hline
Invalid Email Format & Verify that the system detects and rejects invalid email formats. \\
\hline
Invitaion to an Email without Notion & Verify that the system detects when the given email address has no associated Notion account. \\
\hline
\end{longtable}\subsection{\textbf{Use Case: Manage Teamspace Permissions}}
\textbf{Scenario:} A teamspace owner may change the permission levels of any other owner or member.A member may only change the permission levels of other members.\\
\textbf{Actors:} Teamspace Owner, Teamspace Member\\
\textbf{Steps:}
\begin{enumerate}
\item Teamspace owner changes the permission level of other members to one of 'Full Access', 'Çan View', 'Can Edit' or 'Can Comment'
\item Alternatively, Teamspace owner upgrades another member to owner.
\item Alternatively, Teamspace owner downgrades another owner to member.
\end{enumerate}
\textbf{Test Cases}

            \begin{longtable}{|p{0.3\textwidth}|p{0.6\textwidth}|}
            \hline
            \textbf{Name} & \textbf{Description} \\
            \hline
            Consistent Permissions & Verify that consistent teamspace permissions can be set properly. \\
\hline
Invalid Permission Level & Verify that the permission level falls under one of the 4 categories. \\
\hline
Permission Level Update by Member & Verify that a general member cannot alter the permission levels of all members. \\
\hline
Downgrade of Owner by Member & Verify that a general member cannot downgrade a owner to a general member. \\
\hline
\end{longtable}\subsection{\textbf{Use Case: Share Page with Individuals}}
\textbf{Scenario:} A user shares his page with other members or groups while specifying the access levels of each.\\
\textbf{Actors:} User\\
\textbf{Steps:}
\begin{enumerate}
\item User selects a page from his workspace.
\item User selects names of members or groups to share with.
\item User sets permission levels of each of the members or groups.
\end{enumerate}
\textbf{Test Cases}

            \begin{longtable}{|p{0.3\textwidth}|p{0.6\textwidth}|}
            \hline
            \textbf{Name} & \textbf{Description} \\
            \hline
            Valid Sharing & Verify that the page is shared properly when invites members and groups are valid. \\
\hline
Sharing with only Members & Verify that the page sharing is successful when only members are added and no groups are added. \\
\hline
Sharing with only Groups & Verify that the page sharing is successful when only groups are added and no individual members are added. \\
\hline
Empty Sharing Specification & Verify that the page sharing is rejected when no members or groups are selected. \\
\hline
Invalid Permission Category & Verify that the page sharing is rejected when the chosen permission category does not fall under the allowed 4 types of permissions. \\
\hline
\end{longtable}\subsection{\textbf{Use Case: Publish Page as Website}}
\textbf{Scenario:} A user publishes a page as a website that if publicly available. The user selects domain name, SEO and link settings based on his subscription plan.\\
\textbf{Actors:} User\\
\textbf{Steps:}
\begin{enumerate}
\item User opens a page from his workspace.
\item User proceeds to publish as Website option.
\item User approves the proposed randomly generated website domain.
\item Alternatively, the user may choose his own domain if he is in a PLus subscription
\item The user chooses SEO and LInk Expiration related preferences and finalizes publishing.
\end{enumerate}
\textbf{Test Cases}

            \begin{longtable}{|p{0.3\textwidth}|p{0.6\textwidth}|}
            \hline
            \textbf{Name} & \textbf{Description} \\
            \hline
            Valid Publishing & Verify that the website is created properly when the user follows a valid process. \\
\hline
Choosing Custom Domain in Free Plan & Verify that the cannot choose a custom domain when he is in a Free plan. \\
\hline
Choosing Custom Domain in Plus Plan & Verify that the can choose a custom domain when he is in a Plus subscription plan. \\
\hline
Specifying Link Expiry Period in Free Plan & Verify that the cannot specify a link expiry period when he is in a Free plan. \\
\hline
\end{longtable}\subsection{\textbf{Use Case: Collaborate in real-time}}
\textbf{Scenario:} One or more than one user collaborates on a page content in real-time. The system approves or rejects the edits based on consistency.\\
\textbf{Actors:} User\\
\textbf{Steps:}
\begin{enumerate}
\item User opens tha shared page.
\item User selects a particular block of the page.
\item User edits the content of the selected block.
\end{enumerate}
\textbf{Test Cases}

            \begin{longtable}{|p{0.3\textwidth}|p{0.6\textwidth}|}
            \hline
            \textbf{Name} & \textbf{Description} \\
            \hline
            Valid Editing & Verify that the editing is successful when all editors have enough privilege and they are not simultaneously modifying the same block. \\
\hline
Insufficient Permission & Verify that the editing is discarded when some editor has insufficient access privilege. \\
\hline
Simultaneous Editing & Verify that the editing is discarded when more than one editor are trying to simultaneously modify the same block. \\
\hline
\end{longtable}\subsection{\textbf{Use Case: Add comments to Page Content}}
\textbf{Scenario:} On the comment portion of a page block, user writes a text and/or mentions another persons/guests/pages.\\
\textbf{Actors:} User\\
\textbf{Steps:}
\begin{enumerate}
\item User selects a particular block of the page.
\item User writes a textual comment.
\item Alternatively, user may mention other persons or groups or pages.
\end{enumerate}
\textbf{Test Cases}

            \begin{longtable}{|p{0.3\textwidth}|p{0.6\textwidth}|}
            \hline
            \textbf{Name} & \textbf{Description} \\
            \hline
            Valid Comment & Verify that the comment is successfully posted when all parameters are correct. \\
\hline
Self Mentioning & Verify that the comment is successfully posted without sending notification when a user mentions himself/herself. \\
\hline
Empty Text and Mention & Verify that the comment is rejected if no text or mention is there. \\
\hline
Only Mentioning without Text & Verify that the comment is posted even if it does not have a text but mentions some users/groups. \\
\hline
\end{longtable}\subsection{\textbf{Use Case: Suggest Edits to Content}}
\textbf{Scenario:} A user suggests edit to a block of a page content and the owner accepts or rejects the proposed change.\\
\textbf{Actors:} Page owner, Guest User\\
\textbf{Steps:}
\begin{enumerate}
\item User selects a particular block of the page.
\item User writes a textual comment.
\item Alternatively, user may mention other persons or groups or pages.
\end{enumerate}
\textbf{Test Cases}

            \begin{longtable}{|p{0.3\textwidth}|p{0.6\textwidth}|}
            \hline
            \textbf{Name} & \textbf{Description} \\
            \hline
            Acceptance by Owner & Verify that the edit is properly reflected on the content when the owner approves the change. \\
\hline
Rejection by Owner & Verify that the edit is discarded when the owner rejects the change. \\
\hline
\end{longtable}\section{\textbf{Plans and Payment}}
\subsection{\textbf{Use Case: Upgrade plan}}
\textbf{Scenario:} A user wants to upgrades their subscription plan to access additional features and benefits.\\
\textbf{Actors:} User\\
\textbf{Steps:}
\begin{enumerate}
\item User navigates to the Upgrade plan or Explore plans section in the workspace settings
\item User selects the desired plan to upgrade to
\item User inputs the payment details
\item User confirms the upgrade
\item The system processes the payment and upgrades the plan
\item User receives a confirmation message along with an email
\end{enumerate}
\textbf{Test Cases}

            \begin{longtable}{|p{0.3\textwidth}|p{0.6\textwidth}|}
            \hline
            \textbf{Name} & \textbf{Description} \\
            \hline
            User is not an admin or workspace owner & Verify that only an admin or workspace owner can upgrade the plan \\
\hline
Valid payment details & Verify the payment information provided by the user can yield the payment \\
\hline
Invalid payment details & Verify that the payment information provided by the user is invalid and cannot be used for payment \\
\hline
Sufficient balance & Verify that the user has sufficient balance to upgrade the plan \\
\hline
Insufficient balance & Verify that the user does not have sufficient balance to upgrade the plan \\
\hline
Successful upgrade & Verify that the plan is successfully upgraded after payment processing \\
\hline
\end{longtable}\subsection{\textbf{Use Case: Downgrade plan}}
\textbf{Scenario:} A user wants to downgrade their subscription plan to reduce costs or access fewer features\\
\textbf{Actors:} User\\
\textbf{Steps:}
\begin{enumerate}
\item User navigates to the Plans section in the workspace settings
\item User selects the desired plan to downgrade to
\item User confirms the downgrade
\item The system processes the request and adjusts the plan
\item User receives a confirmation message along with an email
\end{enumerate}
\textbf{Test Cases}

            \begin{longtable}{|p{0.3\textwidth}|p{0.6\textwidth}|}
            \hline
            \textbf{Name} & \textbf{Description} \\
            \hline
            User is not an admin or workspace owner & Verify that only an admin or workspace owner can upgrade the plan \\
\hline
Successful downgrade & Verify that the plan is successfully downgraded after the user confirms the downgrade \\
\hline
\end{longtable}\subsection{\textbf{Use Case: Update payment method}}
\textbf{Scenario:} A user wants to update their payment method for the subscription plan\\
\textbf{Actors:} User\\
\textbf{Steps:}
\begin{enumerate}
\item User navigates to the Billing section in the workspace settings
\item User modifies the details in the Payment method section
\item User confirms the update
\item The system processes the request and updates the payment method
\item User receives a confirmation message along with an email
\end{enumerate}
\textbf{Test Cases}

            \begin{longtable}{|p{0.3\textwidth}|p{0.6\textwidth}|}
            \hline
            \textbf{Name} & \textbf{Description} \\
            \hline
            User is not an admin or workspace owner & Verify that only an admin or workspace owner can upgrade the plan \\
\hline
Valid payment method & Verify that the new payment method provided by the user is valid and can be used for future payments \\
\hline
Successful payment method update & Verify that the payment method is successfully updated after the user confirms the update \\
\hline
\end{longtable}\section{\textbf{Third Party Integration}}
\subsection{\textbf{Use Case: Task Management Integration}}
\textbf{Scenario:} A third-party task management application integrates with Notion to synchronize tasks across platforms.\\
\textbf{Actors:} User, Notion API, Third-Party Task Management System\\
\textbf{Steps:}
\begin{enumerate}
\item User creates or updates tasks in the third-party app
\item Tasks are synchronized with Notion
\item Notion database reflects updated task information
\end{enumerate}
\textbf{Test Cases}

            \begin{longtable}{|p{0.3\textwidth}|p{0.6\textwidth}|}
            \hline
            \textbf{Name} & \textbf{Description} \\
            \hline
            Successful Task Synchronization & Verify that tasks created in the third-party app are successfully synced to the Notion database. \\
\hline
Task Deletion & Verify that tasks deleted in the third-party app are removed from the Notion database. \\
\hline
API Rate Limit Handling & Verify that the system correctly handles API rate limits when syncing a large number of tasks to Notion. \\
\hline
Unauthorized API Call & Verify that unauthorized API calls are blocked by Notion during task sync attempts. \\
\hline
\end{longtable}\subsection{\textbf{Use Case: CRM Data Synchronization}}
\textbf{Scenario:} A CRM system integrates with Notion to sync customer data and notes to Notion databases.\\
\textbf{Actors:} Sales Representative, Notion API, CRM System\\
\textbf{Steps:}
\begin{enumerate}
\item Sales rep adds or updates customer data in the CRM
\item The system syncs customer data to the Notion database
\item Notion database reflects updated customer information
\end{enumerate}
\textbf{Test Cases}

            \begin{longtable}{|p{0.3\textwidth}|p{0.6\textwidth}|}
            \hline
            \textbf{Name} & \textbf{Description} \\
            \hline
            Customer Data Sync Success & Verify that the customer data is correctly synchronized from the CRM system to Notion. \\
\hline
Duplicate Data Handling & Verify that duplicate customer data entries are properly handled by the integration. \\
\hline
Sync Failure Due to Invalid Data & Verify that invalid data (e.g., missing required fields) causes the sync to fail. \\
\hline
\end{longtable}\end{document}